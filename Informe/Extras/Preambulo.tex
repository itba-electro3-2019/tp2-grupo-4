%Defino el papel y caracteristicas del informe
\documentclass[12pt,a4paper]{article}
\usepackage[a4paper, total={6in, 8in}]{geometry}
\setlength{\parindent}{20pt}				%cuanta sangria al principio de un parrafo
\usepackage{graphicx}						% importar imagenes
\usepackage{indentfirst}					%pone sangria al primer parrafo de una seccion
\usepackage{float} 							% posicion H para floats
\usepackage[colorinlistoftodos]{todonotes}	%permite poner notas con todonotes
\usepackage{mathtools}  					%paquete que simplifica matemática y ecuaciones en LATEX
\usepackage{bm}
\usepackage[utf8]{inputenc}					% caracteres utf8 (tildes, enie) sin tener que usar comandos
\setcounter{secnumdepth}{0}					%Quito la enumeracion de las secciones
%Defino idioma del documento más unos extras
\usepackage[spanish, es-tabla, es-nodecimaldot]{babel}
% texto automatico en español,"tabla" en vez de "cuadro" y no reemplaza puntos decimales por comas

%%%%%%%%%%%%%%%%%%%%%%%%%%%%%%%%%
%%%%%%%% PAQUETES EXTRA %%%%%%%%%
%%%%%%%%%%%%%%%%%%%%%%%%%%%%%%%%%

%Para manejo de figuraaas
\usepackage[export]{adjustbox}
\usepackage{subcaption}

%Para letras de los captions
\usepackage[font=scriptsize,labelfont=bf]{caption}

%Slash en tablas
\usepackage{diagbox}