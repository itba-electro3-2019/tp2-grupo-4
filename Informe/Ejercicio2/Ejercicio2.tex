\section{Ejercicio 2}
Vamos a desarrollar la compatibilidad entre compuertas NOR de las tecnologías LS, HC y HCT.

\subsection{Introducción a las tecnologías}
La tecnología LS es una aplicación de la anteriormente mencionada TTL pero utiliza transistores Schottky, en otras palabras usamos transistores con diodos Schottky.
Mientras que la tecnología HC es una aplicación de lógica MOS, en la cual se utilizan MOS de alta velocidad, HC="High-speed CMOS". Cabe mencionar que esta es incompatible con la TTL.
Por último, HCT es igual a la HC pero con una mejora en compatibilidad para la interconexión con tecnología TTL.
Se tiene que las distintos tipos de compuertas tiene utilizadas tiene las siguientes características:
\begin{table}[H]
	\centering
	\begin{tabular}{|c|c|c|c|}
		\hline
		\diagbox{Parámetros}{Tecnologías} & LS & HC & HCT\\
		\hline
		$V_{IL_{max}}$ & 0.8 V & 1.35 V & 0.8 V\\
		\hline
		$V_{IH_{min}}$ & 2 V & 3.2 V & 2 V\\
		\hline
		$V_{OL_{max}}$ & 0.5 V & 0.26 V & 0.26 V\\
		\hline
		$V_{OH_{min}}$ & 2.7 V & 3.98 V & 3.98 V\\
		\hline
		Output current & 8 mA & 4 mA& 4 mA\\
		\hline
		Input current & 0.1 mA & 1 $\mu A$ & 1 $\mu A$\\
		\hline
	\end{tabular}
\end{table}

\subsection{Compatibilidad entre LS y HC}
Se conecto la salida del LS al HC y luego la salida del HC al LS. Los resultados obtenidos fueron:
\begin{table}[H]
	\centering
	\begin{tabular}{|c|c|c|}
		\hline
		\diagbox{Parámetros}{Conexión} & LS-HC & HC-LS\\
		\hline
		$V_{IL_{max}}$ & & \\
		\hline
		$V_{IH_{min}}$ & & \\
		\hline
		$V_{OL_{max}}$ & & \\
		\hline
		$V_{OH_{min}}$ & & \\
		\hline
		$NM_{L}$ & & \\
		\hline
		$NM_{H}$ & & \\
		\hline
	\end{tabular}
\end{table}

%%Falta conclusion

\subsection{Compatibilidad entre LS y HCT}
Se conecto la salida del LS al HCT y luego la salida del HCT al LS. Los resultados obtenidos fueron:
\begin{table}[H]
	\centering
	\begin{tabular}{|c|c|c|}
		\hline
		\diagbox{Parámetros}{Conexión} & LS-HC & HC-LS\\
		\hline
		$V_{IL_{max}}$ & & \\
		\hline
		$V_{IH_{min}}$ & & \\
		\hline
		$V_{OL_{max}}$ & & \\
		\hline
		$V_{OH_{min}}$ & & \\
		\hline
		$NM_{L}$ & & \\
		\hline
		$NM_{H}$ & & \\
		\hline
	\end{tabular}
\end{table}

%%Falta conclusion

\subsection{Fanout}
El fanout es un parámetro que indica la cantidad de compuertas lógicas de una misma tecnología que pueden conectarse en paralelo a la salida de otra compuerta, determinar esté número es esencial debido a que si uno se pasa de está, la tensión de salida, así como el comportamiento de las compuertas a la salida de esta puede dejar de ser el deseado. De la información obtenida podemos concluir que 
%%%Falta concluir respecto al numero de Fanout junto con el tema compatibilidades