\section{Ejercicio 2}
Vamos a desarrollar la compatibilidad entre compuertas NOR de las tecnologías LS, HC y HCT.

\subsection{Introducción a las tecnologías}
La tecnología LS es una aplicación de la anteriormente mencionada TTL pero utiliza transistores Schottky, en otras palabras usamos transistores con diodos Schottky.
Mientras que la tecnología HC es una aplicación de lógica MOS, en la cual se utilizan MOS de alta velocidad, HC="High-speed CMOS". Cabe mencionar que esta es incompatible con la TTL.
Por último, HCT es igual a la HC pero con una mejora en compatibilidad para la interconexión con tecnología TTL.
Se tiene que las distintos tipos de compuertas tiene utilizadas tiene las siguientes características:
\begin{table}[H]
	\centering
	\begin{tabular}{|c|c|c|c|}
		\hline
		\diagbox{Parámetros}{Tecnologías} & LS & HC & HCT\\
		\hline
		$V_{IL_{max}}$ & 0.8 V & 1.35 V & 0.8 V\\
		\hline
		$V_{IH_{min}}$ & 2 V & 3.2 V & 2 V\\
		\hline
		$V_{OL_{max}}$ & 0.5 V & 0.26 V & 0.26 V\\
		\hline
		$V_{OH_{min}}$ & 2.7 V & 3.98 V & 3.98 V\\
		\hline
		Output current & 8 mA & 4 mA& 4 mA\\
		\hline
		Input current & 0.1 mA & 1 $\mu A$ & 1 $\mu A$\\
		\hline
	\end{tabular}
\end{table}

\subsection{Compatibilidad entre LS y HC}
Se conecto la salida del LS al HC y luego la salida del HC al LS. Los resultados obtenidos del rango de entrada invalida fueron:
\begin{table}[H]
	\centering
	\begin{tabular}{|c|c|c|}
		\hline
		\diagbox{Parámetro}{Conexión} & LS-HC & HC-LS\\
		Entrada invalida[V] & 0.7 a 0.8 & 2.03 a 2.15\\
		\hline
	\end{tabular}
\end{table}

Podemos concluir que para el caso del LS-HC el problema presentado es que el rango de entrada invalida se encuentra dentro del rango de entrada interpretada como 0 por el LS, lo cual implica que hay un problema de incompatibilidad. Mientras que en el caso HC-LS, esté se encuentra dentro de la zona prohibida de la compuerta HC y por lo tanto es valido.

\subsection{Compatibilidad entre LS y HCT}
Se conecto la salida del LS al HCT y luego la salida del HCT al LS. Los resultados obtenidos del rango de entrada invalida fueron:
\begin{table}[H]
	\centering
	\begin{tabular}{|c|c|c|}
		\hline
		\diagbox{Parámetro}{Conexión} & LS-HC & HC-LS\\
		Entrada invalida[V] & 0.84 a 0.92 & 0.83 a 0.93\\
		\hline
	\end{tabular}
\end{table}

No se encuentran problemas dado que tanto para la conexión LS-HCT y HCT-LS están dentro del rango de zona prohibida de la primera compuerta enchufada.

\subsection{Fanout}
El fanout es un parámetro que indica la cantidad de compuertas lógicas de una misma tecnología que pueden conectarse en paralelo a la salida de otra compuerta, determinar esté número es esencial debido a que si uno se pasa de está, la tensión de salida, así como el comportamiento de las compuertas a la salida de esta puede dejar de ser el deseado. De la información obtenida podemos concluir que gracias a su al bajo valor del input current obtendremos un alto fanout siempre que se utilicen las familias HC y HCT, al mismo tiempo debido a que la familia LS posee el mayor output current está nos va a permitir un fanout de 8000 si se realiza LS-HC o LS-HCT mientras que un HC-LS o HCT-LS nos da un fanout de 40.
Se terminan recomendando siempre conectar compuertas de la misma familia en caso contrario interconectar LS con HCT y de utilizar muchas compuertas en paralelo preferentemente utilizar una conexión LS-HCT.