\documentclass{standalone}

\usepackage[american,nooldvoltagedirection,siunitx]{circuitikz}
\usepackage{tikz}
\usepackage{pbox}

\newcommand{\ctikzlabel}[2]{\pbox{\textwidth}{#1\\#2}} % multiple-lines labels
\usepackage{relsize}
\tikzset{
    pin/.style = {font = \relsize{-2}} % pin font size
}
\ctikzset{
    bipoles/length = 2em, % bipole size
    font = \relsize{-1}, % default font size
}

\begin{document}



\begin{circuitikz}[scale=1]
% U1 NE555
\draw [thick] (5.5,2) coordinate (u1) rectangle ++(2,3); % shape
\draw [pin] (u1) ++ (0,0.5) coordinate (u1 con)
    node[right]{CON}
    node[above left]{5}; % CON
\draw [pin] (u1) ++ (0,1) coordinate (u1 tri)
    node[right]{TRI}
    node[above left]{2}; % TRI
\draw [pin] (u1) ++ (0,1.5) coordinate (u1 thr)
    node[right]{THR}
    node[above left]{6}; % THR
\draw [pin] (u1) ++ (0,2) coordinate (u1 dis)
    node[right]{DIS}
    node[above left]{7}; % DIS
\draw [pin] (u1) ++ (0,2.5) coordinate (u1 rst) 
    node[right]{RST}
    node[above left]{4}; % RST
\draw [pin] (u1) ++ (1,3) coordinate (u1 vcc)
    node[below]{VCC}
    node[above left]{8}; % VCC
\draw [pin] (u1) ++ (1,0) coordinate (u1 gnd)
    node[above]{GND}
    node[below left]{1}; % GND
\draw [pin] (u1) ++ (2,2.5) coordinate (u1 out)
    node[left]{OUT}
    node[above right]{3}; % OUT
\draw (u1) ++ (2,0)
    node[right]{\ctikzlabel{$U_1$}{NE555}}; % NE555P

% U1 NE555 Pins
\draw (u1 con) -- ++ (-1,0); % CON
\draw (u1 tri) -- ++ (-1.5,0); % TRI
\draw (u1 thr) -- ++ (-1,0); % THR
\draw (u1 dis) -- ++ (-1.5,0); % DIS
\draw (u1 rst) -- ++ (-1,0); % RST
\draw (u1 vcc) -- ++ (0,0.5); % VCC
\draw (u1 gnd) -- ++ (0,-1); % GND
\draw (u1 out) -- ++ (1,0); % OUT

\draw (u1 con) ++ (-1,-1.5) node[](gnd){};  

\draw (u1 vcc) ++ (0,0.5) node[](Vc){};  
\draw (u1 vcc) -- (Vc.center);
\draw (u1 gnd) ++ (0,-1) node[sground](gnd1){}; 

%Resto del circuito
\draw (u1 rst) ++ (-1,0) |- (Vc.center); % RST
\draw (Vc)  node[above right]{$5 \,V$};


\draw (u1 tri) ++ (-1,0) node[circ](tri){}; % TRI
\draw (u1 thr) ++ (-1,0) -- (tri);
%\draw (u1 thr) ++ (-1,0) node[circ]{}; % THR

\draw (u1 con) ++ (-1,0) to[C,l=$0.1\,uF$,-*] (gnd);
\draw (gnd.center) -- (gnd1.center);

\draw (u1 thr) -- (u1 tri);
\draw (u1 thr) -- (u1 tri) ++ (-1.5,0) -- ++ (0,-1) -- ++ (-1,0);
\draw (gnd.center) -- ++ (-1.5,0);
\draw (u1 tri) ++ (-2.5,-1) to[C,l=$10\, nF$] (3,1);


\draw (u1 tri) ++ (-2.5,-1) to[R,l=$33\,K\Omega$,*-*] (3,3.5);
\draw (3,3.5) to[R,l=$5K6\,\Omega$] (3,5);
\draw (3,5) |- ++ (1.5,0.5);

\draw (u1 dis) ++ (-1.5,0) -- ++ (0,-0.5) -- ++ (-1,0); % DIS
%nodes
 \draw (Vc) node[ocirc]{}; % THR
  \draw (Vc) node[ocirc]{}; % THR
  \draw (3,5) ++ (1.5,0.5) node[circ]{};
  \draw (gnd1) node[ocirc]{}; % THR
  \draw (u1 out) ++ (1,0) node[label={[font=\footnotesize]right:$Clock$}]{};; % OUT








\end{circuitikz}

 
\end{document}