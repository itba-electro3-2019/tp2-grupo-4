\documentclass{standalone}

\usepackage[american,nooldvoltagedirection,siunitx]{circuitikz}
\usepackage{tikz}
\usepackage{pbox}

\newcommand{\ctikzlabel}[2]{\pbox{\textwidth}{#1\\#2}} % multiple-lines labels
\usepackage{relsize}
\tikzset{
    pin/.style = {font = \relsize{-2}} % pin font size
}
\ctikzset{
    bipoles/length = 2em, % bipole size
    font = \relsize{-1}, % default font size
}

\begin{document}



\begin{circuitikz}[scale=1]
% U1 NE555
\draw [thick] (5.5,2) coordinate (u1) rectangle ++(2,3); % shape
\draw [pin] (u1) ++ (0,0.5) coordinate (u1 con)
    node[right]{CON}
    node[above left]{5}; % CON
\draw [pin] (u1) ++ (0,1) coordinate (u1 tri)
    node[right]{TRI}
    node[above left]{2}; % TRI
\draw [pin] (u1) ++ (0,1.5) coordinate (u1 thr)
    node[right]{THR}
    node[above left]{6}; % THR
\draw [pin] (u1) ++ (0,2) coordinate (u1 dis)
    node[right]{DIS}
    node[above left]{7}; % DIS
\draw [pin] (u1) ++ (0,2.5) coordinate (u1 rst) 
    node[right]{RST}
    node[above left]{4}; % RST
\draw [pin] (u1) ++ (1,3) coordinate (u1 vcc)
    node[below]{VCC}
    node[above left]{8}; % VCC
\draw [pin] (u1) ++ (1,0) coordinate (u1 gnd)
    node[above]{GND}
    node[below left]{1}; % GND
\draw [pin] (u1) ++ (2,2.5) coordinate (u1 out)
    node[left]{OUT}
    node[above right]{3}; % OUT
\draw (u1) ++ (2,0)
    node[right]{\ctikzlabel{$NE$}{$555$}}; % NE555P

% U1 NE555 Pins
\draw (u1 con) -- ++ (-1,0); % CON
\draw (u1 tri) -- ++ (-1.5,0); % TRI
\draw (u1 thr) -- ++ (-1.5,0); % THR
\draw (u1 dis) -- ++ (-1.5,0); % DIS
\draw (u1 rst) -- ++ (-1,0); % RST
\draw (u1 vcc) -- ++ (0,1); % VCC
\draw (u1 gnd) -- ++ (0,-1); % GND
\draw (u1 out) -- ++ (1,0); % OUT

%Resto del circuito
\draw (0,6) node[above left]{$V_{CC}$};
\draw (0,0) node[below left]{$GND$};
\draw (2,6) to[C,l=$1\,pF$,*-] (2,5);
\draw (2,5) node[sground]{};

\draw (0,6) -| (u1 vcc);
\draw (0,0) -| (u1 gnd);
\draw (u1 out) -- ++ (1,0) -- (8.5,0);
\draw (8.5,0)  -- (0,0);
\draw (u1 con) -- ++ (-1,0) to[C,l=$0.1\,uF$,-*] (4.5,0);
\draw (u1 rst) ++ (-1,0) -- (4.5,6); % RST

\draw (11,4) node[pnp](pnp){$2N2907$}; % THR
\draw (u1 dis) ++ (-1.5,0) -- ++ (0,-2.5) -- ++ (5,0) -| (pnp.C); % DIS
\draw (pnp.E) to[R, a_=$R_1$,l=$2\,K\Omega$] (11,6) -- (0,6);
\draw (pnp.B) -- ++ (-1.5,0) to[R,a=$R_2$,l=$10K\,\Omega$,-*] (9.09,6);
\draw (pnp.B) -- ++ (-1.5,0) to[R,a=$R_3$,l=$20K\Omega$] (9.09,2);
\draw (9.09,2) node[sground]{};
\draw (10,1.5) to[C, a=$C_1$,l=$22\,uF$,*-] (10,0) -- (0,0);
\draw (10,1.5) -- ++ (2,0);


%node
 \draw (12,1.5) node[circ,label={[font=\footnotesize]above:$Rampa$}]{};
 \draw (4.5,6) node[circ]{};
 \draw (6.5,0) node[circ]{};
 \draw (8.5,0) node[circ]{};
 \draw (0,6) node[circ](Vcc){};
 \draw (pnp.B) ++ (-1.5,0) node[circ]{};
 \draw (11,1.5) node[circ]{};
 \draw (0,0) node[circ]{};
 \draw (6.5,6) node[circ]{};
 \draw (u1 tri) ++ (-1.5,0) node[circ]{}; % TRI
 \draw (u1 thr) ++ (-1.5,0) node[circ]{}; % THR





\end{circuitikz}

 
\end{document}