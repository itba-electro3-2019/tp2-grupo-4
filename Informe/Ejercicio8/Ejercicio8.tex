\section{Ejercicio 8}
Se decidió separar el diseño en varios bloques que cumplieran tareas específicas, facilitando entre otras la implementación y el testo. Los bloques utilizados fueron un generador de rampa, un comparador, un generador de clock y un contador. Su propuesta de diseño e implementación se presentan a continuación. \par

\subsection{Diseño}

\subsubsection{Generador de Rampa}

Primeramente, se necesitó medir de cierta manera la variación en la tensión provista por el joystick, la cual será proporcional a su posición respecto del eje. Esto se debe a que internamente el joystick cuenta con un ponteciómetro que provee entre $0\,V$ y $5\,V$. Por ende, se diseñó una rampa utilizando un circuito integrado $NE555$, la misma se utilizará para comparar tensiones. \par

\begin{figure}[H]
\centering
\includegraphics[scale=0.8]{Ejercicio8/Circuitos/Generador_de_rampa.pdf}
\caption{Generador de Rampa con desfasaje}
\label{fig:Generador_de_rampa}
\end{figure}

La rampa representará la carga del capacitor $C_1$ presente en el circuito (\ref{fig:Generador_de_rampa}), dicha carga será constante debido a que el componente anteriormente mencionado está conectado con el colector del transistor PNP que está actuando como fuente de corriente. La tasa de refresco estará dada por la frecuencia de la rampa, debido a que la cátedra solicitó una tasa de entre 1 y 20 segundos, fue necesario añadir componentes variables al circuito. Para lograr dicha variación, se cambiará la pendiente de la misma, dada por $S=\frac{I_C}{C_1}$. Para comenzar, se decidió fijar el valor de $C_1$ en $220\,pf$ tal que sólo sea variable $I_C$. Por otro lado, la corriente en el colector está dada por:

\begin{equation}
I_C=\frac{V_{CC}-V_E}{R_E}
\end{equation}

La tensión en el emisor está determinada por un divisor resistivo dado por:

\begin{equation}
V_E=\frac{R_4}{R_4+R_3} * V_{CC} + V_{BE}
\end{equation}

Se fijaron los valores de $R_4$ y $R_3$ en $20K\,\Omega$ y $10K\,\Omega$ respectivamente, tal que la única incógnita sea $R_E$. \par
Para obtener una tasa de refresco de entre 1 y 20 segundos se necesitará que $R_E$ varíe entre $150\,\Omega$ y $4K\,\Omega$. Consecuentemente, se pondrán dos resistencias en serie, siendo la primer resistencia($R_1$), fija con valor nominal $150\,\Omega$ mientras que la segunda($R_2$) será un potenciómetro de $5K\,\Omega$.

Como primer problema a solucionar para el diseño, surgió que la pendiente de la rampa se encontraba entre $5\,V$ y $10\,V$ como se puede ver en la imagen (\ref{fig:Generador_de_rampa_LTSpice}). Esto se debía a la construcción propia del N555 que utiliza comparadores para obtener una señal de salida entre un tercio y dos tercios de la tensión de alimentación ($V_{CC}$), en nuestro caso, $15\,V$.


\begin{figure}[H]
\centering
\includegraphics[width=0.7\textwidth]{Ejercicio8/Imagenes/Rampa_desfasada}
\caption{Tensión de la rampa con desfasaje}
\label{fig:Generador_de_rampa_desfasada_LTSpice}
\end{figure}

Por ende, se implementó un amplificador operacional que funcionará como restador para reducir la tensión de salida de la rampa en $5\,V$, utilizando el siguiente circuito:

\begin{figure}[H]
\centering
\includegraphics[scale=0.8]{Ejercicio8/Circuitos/Restador.pdf}
\caption{Restador}
\label{fig:Restador}
\end{figure}
La salida del amplificador operacional va a estar dada por:
\begin{equation}
V_o=\frac{-R_2}{R_1}V_2+\left(1+\frac{R_4}{R_3}\right)V_1
\end{equation}
Tomando valores de resistencias equivalentes, obtendremos:
\begin{equation}
V_o=-V_2+V_1
\end{equation}
Siendo $V_1$ la tensión de la rampa y $V_2$ la tensión de desfasaje.\par\par
Se obtuvo la siguiente rampa acorde a las necesidades para el trabajo:

\begin{figure}[H]
\centering
\includegraphics[width=0.7\textwidth]{Ejercicio8/Imagenes/Rampa}
\caption{Tensión de la rampa}
\label{fig:Generador_de_rampa_LTSpice}
\end{figure}



Se considera necesario aclarar la utilización de un buffer entre la salida de la rampa desfasada y el restador para que no se modifiquen los comportamientos entre ambos circuitos.\par

\subsubsection{Clock}


\subsubsection{Comparador}

\begin{figure}[H]
\centering
\includegraphics[scale=0.8]{Ejercicio8/Circuitos/Comparador.pdf}
\caption{Comparador de tensiones}
\label{fig:Comparador}
\end{figure}

\subsubsection{Contador}

\begin{figure}[H]
\centering
\includegraphics[scale=0.8]{Ejercicio8/Circuitos/Contador&Display.pdf}
\caption{Contador}
\label{fig:Contador&Display}
\end{figure}

\begin{figure}[H]
\centering
\includegraphics[scale=0.8]{Ejercicio8/Circuitos/Reset.pdf}
\caption{Reset}
\label{fig:Reset}
\end{figure}





\subsection{Conclusiones}